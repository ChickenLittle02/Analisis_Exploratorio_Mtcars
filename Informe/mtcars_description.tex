
\documentclass{article}
\usepackage[utf8]{inputenc}
\usepackage{amsmath}
\usepackage{amsfonts}
\usepackage{amssymb}

% Configuración del título

    
    \title{Informe de Análisis Exploratorio de \texttt{mtcars}}
    \author{
      Melani Forsythe Matos \\
      Daniela Guerrero Álvarez \\
      Rubén Martínez Rojas
    }
    \date{} % Sin fecha para que no aparezca
    
\begin{document}

\maketitle

\newpage % Inserta una nueva página aquí

El conjunto de datos \texttt{mtcars} en R es un conjunto clásico que contiene datos sobre automóviles, y tiene 32 observaciones (filas) de 11 variables (columnas). A continuación, describo cada una de las variables, su significado, el tipo de escala, y si son discretas o continuas:

\begin{enumerate}
    \item \textbf{mpg (Miles per Gallon)}

          \begin{itemize}
              \item \textbf{Descripción:} Consumo de combustible del automóvil en millas por galón.
              \item \textbf{Escala:} Cuantitativa Continua.
              \item \textbf{Significado:} Representa la eficiencia del combustible del automóvil, es decir, cuántas millas puede recorrer el automóvil por cada galón de gasolina.
          \end{itemize}

    \item \textbf{cyl (Cylinders)}

          \begin{itemize}
              \item \textbf{Descripción:} Número de cilindros en el motor del automóvil.
              \item \textbf{Escala:} Cuantitativa Discreta.
              \item \textbf{Significado:} Indica cuántos cilindros tiene el motor. Generalmente, los valores comunes son 4, 6 u 8 cilindros.
          \end{itemize}

    \item \textbf{disp (Displacement)}

          \begin{itemize}
              \item \textbf{Descripción:} Desplazamiento del motor en pulgadas cúbicas.
              \item \textbf{Escala:} Cuantitativa Continua.
              \item \textbf{Significado:} Es el volumen total desplazado por todos los pistones dentro del motor en una sola revolución. Es una medida del tamaño del motor.
          \end{itemize}

    \item \textbf{hp (Horsepower)}

          \begin{itemize}
              \item \textbf{Descripción:} Potencia del motor en caballos de fuerza.
              \item \textbf{Escala:} Es una variable Cuantitativa que puede tener valores Continuos, pero en este caso solo guarda valores Discretos, por tanto la trataremos como Cuantitativa Discreta.
              \item \textbf{Significado:} Mide la potencia del motor, es decir, la capacidad del motor para realizar trabajo en una unidad de tiempo.
          \end{itemize}

    \item \textbf{drat (Rear Axle Ratio)}

          \begin{itemize}
              \item \textbf{Descripción:} Relación de transmisión del eje trasero.
              \item \textbf{Escala:} Cuantitativa Continua.
              \item \textbf{Significado:} Es la relación entre las revoluciones del eje de transmisión y las revoluciones del eje trasero. Afecta el rendimiento y la velocidad del vehículo.
          \end{itemize}

    \item \textbf{wt (Weight)}

          \begin{itemize}
              \item \textbf{Descripción:} Peso del automóvil en miles de libras.
              \item \textbf{Escala:} Cuantitativa Continua.
              \item \textbf{Significado:} El peso del automóvil influye en su eficiencia, aceleración y manejo.
          \end{itemize}

    \item \textbf{qsec (1/4 Mile Time)}

          \begin{itemize}
              \item \textbf{Descripción:} Tiempo en segundos para recorrer un cuarto de milla.
              \item \textbf{Escala:} Cuantitativa Continua.
              \item \textbf{Significado:} Es una medida del tiempo que tarda el automóvil en recorrer un cuarto de milla, comúnmente usado para medir el rendimiento en aceleración.
          \end{itemize}

    \item \textbf{vs (Engine Shape)}

          \begin{itemize}
              \item \textbf{Descripción:} Forma del motor (0 = motor en V, 1 = motor en línea).
              \item \textbf{Escala:}  Esta es de cierta manera una variable cualitativa nominal, lo que esta convertida a CUantitativa Discreta (Binaria).
              \item \textbf{Significado:} Indica el tipo de configuración del motor: si los cilindros están dispuestos en forma de V o en línea.
          \end{itemize}

    \item \textbf{am (Transmission)}

          \begin{itemize}
              \item \textbf{Descripción:} Tipo de transmisión (0 = automática, 1 = manual).
              \item \textbf{Escala:} Esta es de cierta manera una variable cualitativa nominal, lo que esta convertida a CUantitativa Discreta (Binaria).
              \item \textbf{Significado:} Indica si el automóvil tiene una transmisión automática o manual.
          \end{itemize}

    \item \textbf{gear (Gears)}

          \begin{itemize}
              \item \textbf{Descripción:} Número de velocidades de la caja de cambios.
              \item \textbf{Escala:} Cuantitativa Discreta.
              \item \textbf{Significado:} Indica cuántas marchas tiene la caja de cambios del automóvil.
          \end{itemize}

    \item \textbf{carb (Carburetors)}

          \begin{itemize}
              \item \textbf{Descripción:} Número de carburadores.
              \item \textbf{Escala:}Cuantitativa Discreta.
              \item \textbf{Significado:} Indica cuántos carburadores tiene el automóvil, lo que afecta la mezcla de aire y combustible y, por ende, el rendimiento del motor.
          \end{itemize}

\end{enumerate}

\end{document}
